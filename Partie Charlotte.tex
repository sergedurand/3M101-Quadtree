\documentclass{report}

\usepackage[utf8]{inputenc}
\usepackage[T1]{fontenc}
\usepackage[french]{babel}
\usepackage{textcomp}
\usepackage[official]{eurosym}
\usepackage{lmodern}
\usepackage{graphicx}
\usepackage{caption}
\usepackage{listings}
\usepackage{hyperref}
\usepackage{verbatim}
\usepackage{listings}
\usepackage{tikz}
\usetikzlibrary{arrows,positioning, calc}
\tikzstyle{vertex}=[draw,fill=white!15,circle,minimum size=20pt,inner sep=0pt]
\usepackage{color}
\definecolor{darkWhite}{rgb}{0.94,0.94,0.94}
\lstset{
  aboveskip=3mm,
  belowskip=-2mm,
  backgroundcolor=\color{darkWhite},
  basicstyle=\footnotesize,
  breakatwhitespace=false,
  breaklines=true,
  captionpos=b,
  commentstyle=\color{red},
  deletekeywords={...},
  escapeinside={\%*}{*)},
  extendedchars=true,
  framexleftmargin=16pt,
  framextopmargin=3pt,
  framexbottommargin=6pt,
  frame=tb,
  keepspaces=true,
  keywordstyle=\color{blue},
  language=python,
  literate=
  {²}{{\textsuperscript{2}}}1
  {⁴}{{\textsuperscript{4}}}1
  {⁶}{{\textsuperscript{6}}}1
  {⁸}{{\textsuperscript{8}}}1
  {€}{{\euro{}}}1
  {é}{{\'e}}1
  {è}{{\`{e}}}1
  {ê}{{\^{e}}}1
  {ë}{{\¨{e}}}1
  {É}{{\'{E}}}1
  {Ê}{{\^{E}}}1
  {û}{{\^{u}}}1
  {ù}{{\`{u}}}1
  {â}{{\^{a}}}1
  {à}{{\`{a}}}1
  {á}{{\'{a}}}1
  {ã}{{\~{a}}}1
  {Á}{{\'{A}}}1
  {Â}{{\^{A}}}1
  {Ã}{{\~{A}}}1
  {ç}{{\c{c}}}1
  {Ç}{{\c{C}}}1
  {õ}{{\~{o}}}1
  {ó}{{\'{o}}}1
  {ô}{{\^{o}}}1
  {Õ}{{\~{O}}}1
  {Ó}{{\'{O}}}1
  {Ô}{{\^{O}}}1
  {î}{{\^{i}}}1
  {Î}{{\^{I}}}1
  {í}{{\'{i}}}1
  {Í}{{\~{Í}}}1,
  morekeywords={*,...},
  numbers=left,
  numbersep=10pt,
  numberstyle=\tiny\color{black},
  rulecolor=\color{black},
  showspaces=false,
  showstringspaces=false,
  showtabs=false,
  stepnumber=1,
  stringstyle=\color{gray},
  tabsize=4,
  title=\lstname,
}

\begin{document}

\section{Les Arbres Binaires}
\subsection{Description et vocabulaire des arbres binaires}
\subsubsection{Définitions générales sur les arbres}

Un \textit{arbre} est une structure de donnée hiérarchique définie par un nombre fini de nœuds. Chaque \textit{nœud} porte un nom, appelé \textit{étiquette} qui représente sa valeur ou l'information associée. On dit que chaque nœud est relié par une \textit{branche}. 
On nomme des nœuds \textit{parents}, \textit{enfants} ou \textit{fils}, \textit{frères}, \textit{ancêtres} ou \textit{descendants}, les nœuds d'un arbre de manière analogue à un arbre généalogique.
Le \textit{degré} d'un nœud est défini par le nombre de fils qu'il possède. Le degré maximal correspond au degré de l'arbre.
La \textit{taille} d'un arbre est son nombre total de nœuds.
Le \textit{chemin} d'un nœud est une suite de nœuds qu'il faut emprunter pour parcourir l'arbre de la racine au nœud en question. On appelle la \textit{longueur} d'un chemin, le nombre de nœuds empruntés.
La \textit{hauteur} (ou profondeur) d'un arbre est la longueur du chemin le plus long.
On considère la racine de l'arbre comme de niveau 1 puis à chaque génération le niveau augmente de 1.

\subsubsection{Les arbres ordonnés}

On dit qu'un arbre est \textit{ordonné} si tous ses nœuds ont une étiquette supérieure ou égale à celle de chacun de ses enfants s'ils existent. Ainsi, l'étiquette de la racine a la valeur maximale. Pour tout chemin de l'arbre, les étiquettes se succèdent dans un ordre décroissant.
On définit un \textit{sous-arbre} comme un autre arbre formé par un sous-ensemble de nœuds et de branches d'un arbre principal.
On appelle \textit{tas} ou \textit{arbre tassé} un arbre binaire ordonné presque complet. C'est-à-dire que tous les niveaux de l'arbre binaire doivent être remplis, sauf peut-être le dernier qui doit être rempli sur la gauche.

\subsubsection{Les arbres d'expression arithmétiques}

Un \textit{arbre binaire d'expression} est un genre d'arbre binaire utilisé pour représenter, comme son nom l'indique, des expressions. Il existe de types d'expression qu'un arbre binaire peut représenter : algébrique et booléenne.
Les feuilles d'un arbre binaire d'expression sont des quantités (constantes ou variables) numérique dans le cas algébrique et "vrai" ($T$) ou "faux" ($F$) dans le cas booléen.
Les autres nœuds sont des opérateurs : addition ($+$), soustraction ($-$), multiplication ($\times$), division($\div$) et puissance ($\ldots^{\ldots}$) dans le cas algébrique ou des quantificateurs : "et" ($\wedge$), "ou" ($\vee$) et la négation ($\neg$) dans le cas booléen.
Un arbre d'expression est alors évalué en appliquant l'opérateur de la racine au valeurs obtenues en évaluant récursivement les sous-arbres de gauche et de droite.
\begin{figure}
\begin{center}
\begin{tikzpicture}[xscale=1,yscale=1]
\tikzstyle{fleche}=[->,>=latex,thick]
\tikzstyle{noeud}=[fill=white,circle]
\tikzstyle{feuille}=[fill=white,circle]
\def\DistanceInterNiveaux{1}
\def\DistanceInterFeuilles{1}
\def\NiveauA{(-0)*\DistanceInterNiveaux}
\def\NiveauB{(-1)*\DistanceInterNiveaux}
\def\NiveauC{(-2)*\DistanceInterNiveaux}
\def\NiveauD{(-3)*\DistanceInterNiveaux}
\def\InterFeuilles{(1)*\DistanceInterFeuilles}
\node[noeud] (R) at ({(2.5)*\InterFeuilles},{\NiveauA}) {$\times$};
\node[noeud] (Ra) at ({(1.5)*\InterFeuilles},{\NiveauB}) {$\div$};
\node[noeud] (Raa) at ({(0.5)*\InterFeuilles},{\NiveauC}) {$-$};
\node[feuille] (Raaa) at ({(0)*\InterFeuilles},{\NiveauD}) {};
\node[feuille] (Raab) at ({(1)*\InterFeuilles},{\NiveauD}) {$8$};
\node[noeud] (Rab) at ({(2.5)*\InterFeuilles},{\NiveauC}) {$+$};
\node[feuille] (Raba) at ({(2)*\InterFeuilles},{\NiveauD}) {$x$};
\node[feuille] (Rabb) at ({(3)*\InterFeuilles},{\NiveauD}) {$5$};
\node[noeud] (Rb) at ({(4.5)*\InterFeuilles},{\NiveauB}) {$\ldots^{\ldots}$};
\node[feuille] (Rba) at ({(4)*\InterFeuilles},{\NiveauC}) {$4$};
\node[feuille] (Rbb) at ({(5)*\InterFeuilles},{\NiveauC}) {$2$};
\draw[fleche] (R)--(Ra);
\draw[fleche] (Ra)--(Raa);
\draw[fleche] (Raa)--(Raab);
\draw[fleche] (Ra)--(Rab);
\draw[fleche] (Rab)--(Raba);
\draw[fleche] (Rab)--(Rabb);
\draw[fleche] (R)--(Rb);
\draw[fleche] (Rb)--(Rba);
\draw[fleche] (Rb)--(Rbb);
\end{tikzpicture}
\caption{Exemple d'arbre binaire d'expression algébrique} \label{fig:Exemple d'arbre binaire d'expression algébrique}
\end{center}
\end{figure} 

\begin{figure}
\begin{center}
\begin{tikzpicture}[xscale=1,yscale=1]
\tikzstyle{fleche}=[->,>=latex,thick]
\tikzstyle{noeud}=[fill=white,circle]
\tikzstyle{feuille}=[fill=white,circle]
\def\DistanceInterNiveaux{1}
\def\DistanceInterFeuilles{1}
\def\NiveauA{(-0)*\DistanceInterNiveaux}
\def\NiveauB{(-1)*\DistanceInterNiveaux}
\def\NiveauC{(-2)*\DistanceInterNiveaux}
\def\NiveauD{(-3)*\DistanceInterNiveaux}
\def\InterFeuilles{(1)*\DistanceInterFeuilles}
\node[noeud] (R) at ({(2.5)*\InterFeuilles},{\NiveauA}) {$\vee$};
\node[noeud] (Ra) at ({(1.5)*\InterFeuilles},{\NiveauB}) {$\wedge$};
\node[noeud] (Raa) at ({(0.5)*\InterFeuilles},{\NiveauC}) {$\neg$};
\node[feuille] (Raaa) at ({(0)*\InterFeuilles},{\NiveauD}) {$F$};
\node[feuille] (Raab) at ({(1)*\InterFeuilles},{\NiveauD}) {};
\node[noeud] (Rab) at ({(2.5)*\InterFeuilles},{\NiveauC}) {$\vee$};
\node[feuille] (Raba) at ({(2)*\InterFeuilles},{\NiveauD}) {$T$};
\node[feuille] (Rabb) at ({(3)*\InterFeuilles},{\NiveauD}) {$F$};
\node[noeud] (Rb) at ({(4.5)*\InterFeuilles},{\NiveauB}) {$\vee$};
\node[feuille] (Rba) at ({(4)*\InterFeuilles},{\NiveauC}) {$T$};
\node[feuille] (Rbb) at ({(5)*\InterFeuilles},{\NiveauC}) {$F$};
\draw[fleche] (R)--(Ra);
\draw[fleche] (Ra)--(Raa);
\draw[fleche] (Raa)--(Raaa);
\draw[fleche] (Ra)--(Rab);
\draw[fleche] (Rab)--(Raba);
\draw[fleche] (Rab)--(Rabb);
\draw[fleche] (R)--(Rb);
\draw[fleche] (Rb)--(Rba);
\draw[fleche] (Rb)--(Rbb);
\end{tikzpicture}
\caption{Exemple d'arbre binaire d'expression booléen} \label{fig:Exemple d'arbre binaire d'expression booléen}
\end{center} 
\end{figure}

\subsection{Particularité des arbres binaires}

Un arbre binaire peut à la fois ne contenir aucun nœud ou être composé de trois ensembles de nœuds disjoints que l'on appelle :
\begin{itemize}
\item un nœud racine;
\item un sous-arbre de gauche;
\item un sous-arbre de droit.
\end{itemize}

Un arbre binaire qui ne contient pas de nœuds est appelé un arbre vide ou un arbre nul.
Si le sous-arbre de gauche n'est pas vide, sa racine est appelée le fils gauche de la racine de l'arbre entier. Il en va de même pour le sous-arbre de droite.
Par exemple, dans l'arbre (a), le nœud « 2 » est le fils de gauche de la racine de l'arbre, il est aussi la racine du sous-arbre de gauche.
Si un sous-arbre est un arbre nul, on dit alors que le fils est absent ou manquant.
\begin{figure}
\begin{center}
\begin{tikzpicture}[xscale=1,yscale=1]
\tikzstyle{fleche}=[->,>=latex,thick]
\tikzstyle{noeud}=[fill=white,circle]
\tikzstyle{feuille}=[fill=white,circle]
\def\DistanceInterNiveaux{1}
\def\DistanceInterFeuilles{1}
\def\NiveauA{(-0)*\DistanceInterNiveaux}
\def\NiveauB{(-1)*\DistanceInterNiveaux}
\def\NiveauC{(-2)*\DistanceInterNiveaux}
\def\NiveauD{(-3)*\DistanceInterNiveaux}
\def\InterFeuilles{(1)*\DistanceInterFeuilles}
\node[noeud] (R) at ({(2)*\InterFeuilles},{\NiveauA}) {$3$};
\node[noeud] (Ra) at ({(1)*\InterFeuilles},{\NiveauB}) {$2$};
\node[noeud] (Raa) at ({(0.5)*\InterFeuilles},{\NiveauC}) {$1$};
\node[feuille] (Raaa) at ({(0)*\InterFeuilles},{\NiveauD}) {$6$};
\node[feuille] (Raab) at ({(1)*\InterFeuilles},{\NiveauD}) {};
\node[feuille] (Rab) at ({(2)*\InterFeuilles},{\NiveauC}) {$4$};
\node[noeud] (Rb) at ({(3.5)*\InterFeuilles},{\NiveauB}) {$7$};
\node[feuille] (Rba) at ({(3)*\InterFeuilles},{\NiveauC}) {$5$};
\node[feuille] (Rbb) at ({(4)*\InterFeuilles},{\NiveauC}) {};
\draw[fleche] (R)--(Ra);
\draw[fleche] (Ra)--(Raa);
\draw[fleche] (Raa)--(Raaa);
\draw[fleche] (Ra)--(Rab);
\draw[fleche] (R)--(Rb);
\draw[fleche] (Rb)--(Rba);
\end{tikzpicture}
\end{center}
\caption{Arbre (a)} \label{fig:Exemples d'arbres}
\end{figure}

\begin{figure}
\begin{center}
\begin{tikzpicture}[xscale=1,yscale=1]
\tikzstyle{fleche}=[->,>=latex,thick]
\tikzstyle{noeud}=[fill=white,circle]
\tikzstyle{feuille}=[fill=white,circle]
\def\DistanceInterNiveaux{1}
\def\DistanceInterFeuilles{1}
\def\NiveauA{(-0)*\DistanceInterNiveaux}
\def\NiveauB{(-1)*\DistanceInterNiveaux}
\def\NiveauC{(-2)*\DistanceInterNiveaux}
\def\NiveauD{(-3)*\DistanceInterNiveaux}
\def\InterFeuilles{(1)*\DistanceInterFeuilles}
\node[noeud] (R) at ({(2)*\InterFeuilles},{\NiveauA}) {$3$};
\node[noeud] (Ra) at ({(1)*\InterFeuilles},{\NiveauB}) {$2$};
\node[noeud] (Raa) at ({(0.5)*\InterFeuilles},{\NiveauC}) {$1$};
\node[feuille] (Raaa) at ({(0)*\InterFeuilles},{\NiveauD}) {};
\node[feuille] (Raab) at ({(1)*\InterFeuilles},{\NiveauD}) {$6$};
\node[feuille] (Rab) at ({(2)*\InterFeuilles},{\NiveauC}) {$4$};
\node[noeud] (Rb) at ({(3.5)*\InterFeuilles},{\NiveauB}) {$7$};
\node[feuille] (Rba) at ({(3)*\InterFeuilles},{\NiveauC}) {};
\node[feuille] (Rbb) at ({(4)*\InterFeuilles},{\NiveauC}) {$5$};
\draw[fleche] (R)--(Ra);
\draw[fleche] (Ra)--(Raa);
\draw[fleche] (Raa)--(Raab);
\draw[fleche] (Ra)--(Rab);
\draw[fleche] (R)--(Rb);
\draw[fleche] (Rb)--(Rbb);
\end{tikzpicture}
\end{center}
\caption{Arbre (b)} \label{fig:Exemples d'arbres}
\end{figure}

\begin{figure}
\begin{center}
\begin{tikzpicture}[xscale=1,yscale=1]
\tikzstyle{fleche}=[->,>=latex,thick]
\tikzstyle{noeud}=[fill=white,circle]
\tikzstyle{feuille}=[fill=gray,draw]
\def\DistanceInterNiveaux{1}
\def\DistanceInterFeuilles{1}
\def\NiveauA{(-0)*\DistanceInterNiveaux}
\def\NiveauB{(-1)*\DistanceInterNiveaux}
\def\NiveauC{(-2)*\DistanceInterNiveaux}
\def\NiveauD{(-3)*\DistanceInterNiveaux}
\def\NiveauE{(-4)*\DistanceInterNiveaux}
\def\InterFeuilles{(1)*\DistanceInterFeuilles}
\node[noeud] (R) at ({(3.5)*\InterFeuilles},{\NiveauA}) {$3$};
\node[noeud] (Ra) at ({(2)*\InterFeuilles},{\NiveauB}) {$2$};
\node[noeud] (Raa) at ({(1)*\InterFeuilles},{\NiveauC}) {$1$};
\node[noeud] (Raaa) at ({(0.5)*\InterFeuilles},{\NiveauD}) {$6$};
\node[feuille] (Raaaa) at ({(0)*\InterFeuilles},{\NiveauE}) {};
\node[feuille] (Raaab) at ({(1)*\InterFeuilles},{\NiveauE}) {};
\node[feuille] (Raab) at ({(2)*\InterFeuilles},{\NiveauD}) {};
\node[noeud] (Rab) at ({(3.5)*\InterFeuilles},{\NiveauC}) {$4$};
\node[feuille] (Raba) at ({(3)*\InterFeuilles},{\NiveauD}) {};
\node[feuille] (Rabb) at ({(4)*\InterFeuilles},{\NiveauD}) {};
\node[noeud] (Rb) at ({(6)*\InterFeuilles},{\NiveauB}) {$7$};
\node[feuille] (Rba) at ({(5)*\InterFeuilles},{\NiveauC}) {};
\node[noeud] (Rbb) at ({(6.5)*\InterFeuilles},{\NiveauC}) {$5$};
\node[feuille] (Rbba) at ({(6)*\InterFeuilles},{\NiveauD}) {};
\node[feuille] (Rbbb) at ({(7)*\InterFeuilles},{\NiveauD}) {};
\draw[fleche] (R)--(Ra);
\draw[fleche] (Ra)--(Raa);
\draw[fleche] (Raa)--(Raaa);
\draw[fleche] (Raaa)--(Raaaa);
\draw[fleche] (Raaa)--(Raaab);
\draw[fleche] (Raa)--(Raab);
\draw[fleche] (Ra)--(Rab);
\draw[fleche] (Rab)--(Raba);
\draw[fleche] (Rab)--(Rabb);
\draw[fleche] (R)--(Rb);
\draw[fleche] (Rb)--(Rba);
\draw[fleche] (Rb)--(Rbb);
\draw[fleche] (Rbb)--(Rbba);
\draw[fleche] (Rbb)--(Rbbb);
\end{tikzpicture}
\end{center}
\caption{Arbre (c)} \label{fig:Exemples d'arbres}
\end{figure}

Un arbre binaire de recherche n'est pas seulement un arbre ordonné où chaque nœud a un degré maximal de 2.  En effet, pour une racine n'ayant qu'un seul fils, sa position, s'il s'agit du fils de gauche ou du fils de droite, importe contrairement à un arbre ordonné.
La figure 1 représente des arbres binaires dessinés sous leur forme standard. Attention, l'arbre (a) est différent de l'arbre (b) quand ils sont considérés comme des arbres binaires, mais en tant que arbres classiques, ils sont identiques.
Cependant, un arbre binaire peut être associé à un arbre ordonné en ajoutant une représentation explicite des informations manquantes comme dans l'arbre (c). L'idée est de remplacer chaque enfant manquant de l'arbre binaire par un nœud n'ayant aucune descendance. Ces nœuds sont appelés des feuilles et sont représentés par des carrés dans l'arbre (c). On parle aussi de nœud externe pour une feuille en opposition aux nœuds internes. On obtient alors un arbre binaire complet : chaque nœud est soit une feuille soit a un degré égal à 2. Il n'y a pas de nœud de degré 1.


\section{Les arbres binaires de recherche}
\subsection{Affichage}

Pour réaliser les différents parcours décrits ci-dessous, on peut définir un arbre par un dictionnaire qui associe chaque nœud à ses fils. Par exemple, l'arbre (a) est définit par :
\begin{lstlisting}
A = {3:[2, 7], 2:[1, 4], 7:[5], 1:[6], 4:[], 5:[], 6:[]}
\end{lstlisting}
C'est cette méthode qui est utilisée pour décrire un arbre dans le cas du parcours en largeur. Cependant, on peut également définir une classe Arbre telle que :
\begin{lstlisting}
class Arbre: 
    def __init__(self,valeur): 
        self.gauche = None
        self.droit = None
        self.racine = valeur
# L'arbre (a) est donc défini par :
A = Arbre(3) 
A.gauche = Arbre(2) 
A.droit = Arbre(7) 
A.gauche.gauche = Arbre(1) 
A.gauche.droit = Arbre(4)
A.gauche.gauche.gauche = Arbre(6)
A.droit.gauche = Arbre(5)
\end{lstlisting}
On utilisera cette classe pour les parcours en profondeur.
\subsubsection{Parcours en largeur}
Le principe d'un parcours en largeur est de lister les nœuds de l'arbre niveau par niveau en commençant par les nœuds de niveau 1 puis les nœuds de niveau 2 et ainsi de suite. Dans chaque niveau, les nœuds sont parcourus de gauche à droite.
Le parcours en largeur de l'arbre (a) est donc $[3, 2, 7, 1, 4, 5, 6]$.

L'algorithme d'un tel parcours se fait à l'aide d'une file (premier entré, premier sorti). On enfile d'abord la racine. Puis on enfile les fils du premier nœud de la file uniquement s'ils ne sont pas déjà dans la file et s'ils n'y sont pas déjà passés. Enfin on défile : on supprime le premier nœud de la file. Et ainsi de suite jusqu'à ce que la file soit vide.
\begin{lstlisting}
def parcours_largeur(arbre, racine):
	liste = [racine]
	parcours = [racine]
	while liste :
		x = liste.pop(0)
		for fils in arbre[x]:
			if fils in liste : 
				continue
			parcours.append(fils)
			liste.append(fils)
	return parcours
\end{lstlisting}

\subsubsection{Parcours en profondeur}
Le principe d'un parcours en profondeur est de lister les nœuds de l'arbre récursivement à partir de la racine puis les sous-arbres gauches et droits de cette racine et ainsi de suite pour la totalité de l'arbre.
Il existe plusieurs types de parcours en profondeur : Infixe, Suffixe (ou Postfixe) et Préfixe.
\begin{itemize}
\item Infixe

Le parcours infixe consiste à lister les nœuds en partant du sous-arbre gauche puis remonter à sa racine et enfin parcourir le sous-arbre droit.
Le parcours infixe de l'arbre (a) est donc $[6, 1, 2, 4, 3, 5, 7]$.
\begin{lstlisting}
def parcours_infixe(arbre): 
    if arbre: 
        parcours_infixe(arbre.gauche) 
        print(arbre.racine)
        parcours_infixe(arbre.droit) 
\end{lstlisting}

\item Suffixe

Le parcours suffixe consiste quant à lui à lister les nœuds depuis le sous-arbre gauche puis le sous-arbre droit et enfin remonter à la racine.
Le parcours suffixe de l'arbre (a) est donc $[6, 1, 4, 2, 5, 7, 3]$.
\begin{lstlisting}
def parcours_suffixe(arbre): 
     if arbre: 
        parcours_suffixe(arbre.gauche) 
        parcours_suffixe(arbre.droit) 
        print(arbre.racine)
\end{lstlisting}

\item Préfixe

Le parcours préfixe, finalement, consiste à lister les nœuds en commençant par la racine puis le sous-arbre gauche et enfin le sous-arbre droit.
Le parcours préfixe de l'arbre (a) est donc $[3, 2, 1, 6, 4, 7, 5]$.
\begin{lstlisting}
def parcours_prefixe(arbre):
    if arbre:
        print(arbre.racine)
        parcours_prefixe(arbre.gauche)
        parcours_prefixe(arbre.droit)
\end{lstlisting}

\end{itemize}

\subsection{Insertion}


Enumerative Combinatorics: De Richard P. Stanley \\
\url{https://en.wikipedia.org/wiki/Tree_(graph_theory)}\\
Introduction to algorithms Cormen\\
\url{http://mescal.imag.fr/membres/jean-marc.vincent/index.html/ProTer/Algorithmes-Classiques/Heapsort.pdf}\\


%Pour mettre toute la biblio y compris les références pas citées spécifiquement :
% \nocite{*}
% \nocite{*} 
\bibliography{../biblio}


\listoffigures
\end{document}
