\documentclass{report}

\usepackage[utf8]{inputenc}
\usepackage[T1]{fontenc}
\usepackage[french]{babel}
\usepackage{textcomp}
\usepackage[official]{eurosym}
\usepackage{lmodern}
\usepackage{graphicx}
\usepackage{caption}
\usepackage{listings}
\usepackage{hyperref}
\usepackage{verbatim}
\usepackage{listings}
\usepackage{tikz}
\usetikzlibrary{arrows,positioning, calc}
\tikzstyle{vertex}=[draw,fill=white!15,circle,minimum size=20pt,inner sep=0pt]
\usepackage{color}
\definecolor{darkWhite}{rgb}{0.94,0.94,0.94}
\lstset{
  aboveskip=3mm,
  belowskip=-2mm,
  backgroundcolor=\color{darkWhite},
  basicstyle=\footnotesize,
  breakatwhitespace=false,
  breaklines=true,
  captionpos=b,
  commentstyle=\color{red},
  deletekeywords={...},
  escapeinside={\%*}{*)},
  extendedchars=true,
  framexleftmargin=16pt,
  framextopmargin=3pt,
  framexbottommargin=6pt,
  frame=tb,
  keepspaces=true,
  keywordstyle=\color{blue},
  language=python,
  literate=
  {²}{{\textsuperscript{2}}}1
  {⁴}{{\textsuperscript{4}}}1
  {⁶}{{\textsuperscript{6}}}1
  {⁸}{{\textsuperscript{8}}}1
  {€}{{\euro{}}}1
  {é}{{\'e}}1
  {è}{{\`{e}}}1
  {ê}{{\^{e}}}1
  {ë}{{\¨{e}}}1
  {É}{{\'{E}}}1
  {Ê}{{\^{E}}}1
  {û}{{\^{u}}}1
  {ù}{{\`{u}}}1
  {â}{{\^{a}}}1
  {à}{{\`{a}}}1
  {á}{{\'{a}}}1
  {ã}{{\~{a}}}1
  {Á}{{\'{A}}}1
  {Â}{{\^{A}}}1
  {Ã}{{\~{A}}}1
  {ç}{{\c{c}}}1
  {Ç}{{\c{C}}}1
  {õ}{{\~{o}}}1
  {ó}{{\'{o}}}1
  {ô}{{\^{o}}}1
  {Õ}{{\~{O}}}1
  {Ó}{{\'{O}}}1
  {Ô}{{\^{O}}}1
  {î}{{\^{i}}}1
  {Î}{{\^{I}}}1
  {í}{{\'{i}}}1
  {Í}{{\~{Í}}}1,
  morekeywords={*,...},
  numbers=left,
  numbersep=10pt,
  numberstyle=\tiny\color{black},
  rulecolor=\color{black},
  showspaces=false,
  showstringspaces=false,
  showtabs=false,
  stepnumber=1,
  stringstyle=\color{gray},
  tabsize=4,
  title=\lstname,
}
\begin{document}
Implémentation d’un quadtree en Pyhton :
Pour implémenter un quadtree « point », il existe plusieurs méthodes. Celle présentée dans ce rapport nécessite 2 classes :
\begin{itemize}
\item Une classe « point » : ici, nous travaillons en deux dimensions donc un point est représenté par deux coordonnées
\item Une classe « quadtree » : dans un quadtree chaque noeud correspond à un point dans le plan. Les fils d’un point correspond aux quadrants Nord-Ouest, Sud-Ouest, Sud-Est et Nord-Est. Chaque noeud a donc au plus quatre fils.
\end{itemize}

\begin{lstlisting}
class Point() :
	def __init__(self,ID,x,y) :
		self.ID = ID
		self.x = x
		self.y = y
\end{lstlisting}

On définit une fonction pour vérifier si deux points ont les mêmes identifiants : 
\begin{lstlisting}
def is_equal(self, point) :
	return (self.ID == point.ID)
\end{lstlisting}

Pour définir par la suite un quadtree, il est crucial de pouvoir déterminer la position d’un point en fonction d’un autre c’est-à-dire si un point se trouve au Nord-Ouest, au Sud-Ouest, au Sud-Est ou au Nord-Est d’un autre. Evidemment, si un point est égal à un autre, il n'est dans aucune de ces quatre situations. De plus, il est nécessaire de traiter les cas singuliers où un point serait exactement à la frontière entre plusieurs quadrants. On suivra arbitrairement les règles décrites par la figure~\ref{fig_imp_QT}.


\begin{figure}[h]
\centering
\tikzset{every picture/.style={line width=0.75pt}} %set default line width to 0.75pt     
\begin{tikzpicture}[x=0.75pt,y=0.75pt,yscale=-1,xscale=1]
%uncomment if require: \path (0,470); %set diagram left start at 0, and has height of 470
%Straight Lines [id:da22013873563402342] 
\draw    (180.29,149.86) -- (260.6,150.04) ;
%Straight Lines [id:da6526683913061784] 
\draw [color={rgb, 255:red, 155; green, 155; blue, 155 }  ,draw opacity=1 ]   (200.67,89.67) -- (200,209) ;
%Straight Lines [id:da7623472892810397] 
\draw [color={rgb, 255:red, 155; green, 155; blue, 155 }  ,draw opacity=1 ]   (320.67,90.33) -- (320,209.67) ;
%Straight Lines [id:da4415887027397025] 
\draw [color={rgb, 255:red, 155; green, 155; blue, 155 }  ,draw opacity=1 ]   (320,209.67) -- (200,209) ;
%Straight Lines [id:da9527889195393131] 
\draw [color={rgb, 255:red, 155; green, 155; blue, 155 }  ,draw opacity=1 ]   (320.67,90.33) -- (200.67,89.67) ;
%Straight Lines [id:da13492566984177712] 
\draw    (230.29,150) -- (290.5,150) ;
\draw [shift={(290.5,150)}, rotate = 0] [color={rgb, 255:red, 0; green, 0; blue, 0 }  ][fill={rgb, 255:red, 0; green, 0; blue, 0 }  ][line width=0.75]      (0, 0) circle [x radius= 3.35, y radius= 3.35]   ;
\draw [shift={(230.29,150)}, rotate = 0] [color={rgb, 255:red, 0; green, 0; blue, 0 }  ][fill={rgb, 255:red, 0; green, 0; blue, 0 }  ][line width=0.75]      (0, 0) circle [x radius= 3.35, y radius= 3.35]   ;
%Straight Lines [id:da8076164026512465] 
\draw    (260.4,180.4) -- (260.8,120.4) ;
\draw [shift={(260.8,120.4)}, rotate = 270.38] [color={rgb, 255:red, 0; green, 0; blue, 0 }  ][fill={rgb, 255:red, 0; green, 0; blue, 0 }  ][line width=0.75]      (0, 0) circle [x radius= 3.35, y radius= 3.35]   ;
\draw [shift={(260.4,180.4)}, rotate = 270.38] [color={rgb, 255:red, 0; green, 0; blue, 0 }  ][fill={rgb, 255:red, 0; green, 0; blue, 0 }  ][line width=0.75]      (0, 0) circle [x radius= 3.35, y radius= 3.35]   ;
%Shape: Axis 2D [id:dp14184713416836026] 
\draw  (250.6,150.04) -- (350.6,150.04)(260.6,60.4) -- (260.6,160) (343.6,145.04) -- (350.6,150.04) -- (343.6,155.04) (255.6,67.4) -- (260.6,60.4) -- (265.6,67.4)  ;
%Straight Lines [id:da3187137992537872] 
\draw    (260.6,150.4) -- (260.33,230) ;
\draw   (275.13,175) .. controls (278.5,177.78) and (281.88,179.44) .. (285.25,180) .. controls (281.88,180.56) and (278.5,182.22) .. (275.13,185) ;
\draw   (285.24,135.12) .. controls (287.99,131.71) and (289.61,128.31) .. (290.13,124.94) .. controls (290.73,128.3) and (292.43,131.66) .. (295.24,135.01) ;
\draw   (235.23,164.98) .. controls (232.42,168.33) and (230.73,171.7) .. (230.14,175.06) .. controls (229.61,171.69) and (227.98,168.29) .. (225.23,164.89) ;
\draw   (245.27,124.98) .. controls (241.88,122.21) and (238.5,120.57) .. (235.13,120.02) .. controls (238.49,119.45) and (241.87,117.77) .. (245.23,114.98) ;
%Straight Lines [id:da7462190641238284] 
\draw    (236.25,120) -- (260.8,120.4) ;
%Straight Lines [id:da43374123490339933] 
\draw    (260.4,180.4) -- (285.12,180.06) ;
%Straight Lines [id:da8946248849560359] 
\draw    (290.24,125.39) -- (290.5,150) ;
%Straight Lines [id:da7024140092257973] 
\draw    (230.29,150) -- (230.01,174.94) ;
% Text Node
\draw (222,106.33) node  [align=left] {NW};
% Text Node
\draw (302,106.33) node  [align=left] {NE};
% Text Node
\draw (302,197) node  [align=left] {SE};
% Text Node
\draw (222,197) node  [align=left] {SW};
% Text Node
\draw (355.5,153) node   {$x$};
% Text Node
\draw (255.5,52) node   {$y$};
\end{tikzpicture}
\caption{Règles pour la localisation des points situés sur une frontière entre deux quadrants}
\label{fig:imp_QT}
\end{fig}

Par exemple, pour savoir si un point est au Nord-Ouest (north-west) d'un autre :

\begin{lstlisting}
def is_NW(self, point):  
	""" fonction qui renvoie si le point entré en paramètre se situe au nord-ouest du self
         a.is_NW(b) = True --> le point b est au nord-ouest du point a
        """
	return (self.x >= point.x and self.y < point.y)
		
Il en va de même pour les autres quadrants en modifiant les inégalités (larges ou strictes) selon la figure~\ref{fig:imp_QT}



Pour simplifier les algorithmes, on aura également besoin de mesurer la distance euclidienne entre deux points :
\begin{lstlisting}
def distance(self, point) :
	return ((self.x $-$ point.x)$**$2 $+$ (self.y $-$ point.y)$**$2)$**$(1/2)
\end{lstlisting}


On implémente maintenant la classe « quadtree » :
\begin{lstlisting}
class QuadtreePoint() :
	def __init__(self, point, NW, SW, SE, NE)
		self.point = point
		self.NW = NW
		self.SW = SW
		self.SE = SE
		self.NE = NE
\end{lstlisting}

\subsection{Recherche}

La recherche s'implémente de manière très analogue à celle des ABR. La différence réside uniquement dans le nombre de fils qui est doublé. On procède à nouveau en quatre étapes:
\begin{itemize}
	\item on compare avec le point racine, si l'égalité est vérifiée, on le renvoie directement
	\item sinon, on teste si la valeur recherchée est dans le quadrant NW, si elle l'est, on fait donc un appel récursif sur ce dernier
	\item on répète l'étape précédente pour chaque quadrant, jusqu'à atteindre la valeur recherchée
	\item et finalement si on atteint un point n'ayant pas de quadrant et que son identifiant n'est pas l'élément recherché alors l'élément ne se trouve pas dans le quadtree
\end{itemize}

\begin{lstlisting}
    def recherche(self, x):
        """ recherche d'un point x dans le quadtree """
        if self.point.ID is None:
            return "le quadtree est vide"

        if self.point.is_equal(x):
            return self
        else:
            if self.point.is_NW(x):
                if self.NW is not None:
                    return self.NW.recherche(x)
                else:
                    return "le quadrant NW est vide"
            elif self.point.is_SW(x):
                if self.SW is not None:
                    return self.SW.recherche(x)
                else:
                    return "le quadrant SW est vide"
            elif self.point.is_SE(x):
                if self.SE is not None:
                    return self.SE.recherche(x)
                else:
                    return "le quadrant SE est vide"
            elif self.point.is_NE(x):
                if self.NE is not None:
                    return self.NE.recherche(x)
                else:
                    return "le quadrant NE est vide"
\end{lstlisting}

\subsectin{Insertion}

Ici aussi, l'implémentation est très proche de celle des ABR. Pour insérer un nouveau point dans un quadtree, il faut savoir où il doit se situer.  

\begin{lstlisting}
    def insertion(self,x):
        """ insertion d'un point x dans le quadtree """
        if self.point is not None:
            if self.point.is_NW(x):
                if self.NW is None:
                    self.NW = Quadtree(x)
                else:
                    self.NW.insertion(x)
            elif self.point.is_SW(x):
                if self.SW is None:
                    self.SW = Quadtree(x)
                else:
                    self.SW.insertion(x)
            elif self.point.is_SE(x):
                if self.SE is None:
                    self.SE = Quadtree(x)
                else:
                    self.SE.insertion(x)
            elif self.point.is_NE(x):
                if self.NE is None:
                    self.NE = Quadtree(x)
                else:
                    self.NE.insertion(x)
        else:
            self.point = x
\end{lstlisting}

\end{document}
